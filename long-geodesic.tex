\documentclass[oneside,a4paper, 12pt]{article}
\usepackage{anysize}
%\marginsize{left}{right}{top}{bottom}
%\marginsize{2cm}{2cm}{2cm}{2cm}
\usepackage{long-geodesic}

%\usepackage{lineno}\linenumbers

\begin{document}

\title{Long geodesics on convex surfaces}
\author{A. Akopyan and A. Petrunin}
\date{}
\maketitle

\begin{abstract}
We review the theory of intrinsic geometry of convex surfaces in the Euclidean space and prove the following theorem: 
if the surface of convex body $K$ contains arbitrary long closed simple geodesic, then $K$ is an isosceles tetrahedron.
\end{abstract}

\section{Introduction}

The goal of this note is to introduce the reader to the theory of intrinsic geometry of convex surfaces and prove the following theorem.
This theorem is given as an illustration to the theory, not the other way around.

\begin{wrapfigure}{r}{21 mm}
\begin{lpic}[t(-4 mm),b(-0 mm),r(0 mm),l(0 mm)]{pics/long-geodesic(1)}
\end{lpic}
\end{wrapfigure}

Recall that a curve in a convex surface is called geodesic if every sufficiently short arc of the curve is length minimizing
and we say that it is simple if it has no self intersections.

A tetrahedron with equal opposite edges is called \emph{isosceles}.

\begin{thm}{Theorem}
	\label{Long geodesic}
Assume that the surface $\Sigma$ of a convex body $B$ in the Euclidean space $\EE^3$
admits an arbitrary long simple closed geodesic.
Then $B$ is an isosceles tetrahedron.
\end{thm}

A weaker statement was proved by Vladimir Protasov in \cite{protasov}.
 

\section{The theory}

The intrinsic metric on convex surfaces is not arbitrary;
it has certain comparison property, which we about to discuss.
A very short but comprehensive introduction to the subject was written by Alexander Alexandrov in \cite{alexandrov1941}.

By \emph{surface} we will understand a compact two-dimensional surface, possibly with boundary, which is equipped with geodesic metric.
that is, any two points in the surface can be joined by a minimizing geodesic.
A minimizing geodesic connecting two points $p$ and $q$ will be denoted as $[pq]$.

For a triple of points $x,y,z$ in a surface $\Sigma$, a choice of a triple of geodesics $([x y], [y z], [z x])$ will be called a \index{triangle}\emph{triangle} 
and denoted as 
$[x y z]$.
Given a triple $x,y,z\in \Sigma$ there may be many different triangles with these vertices may exist, any of which can be denoted by $[x y z]$.

A triangle $[\~x\~y\~z]$ in the plane $\EE^2$
with the same side lengths as $[x y z]$ 
is called \emph{model triangle} of $[x y z]$;
it can be written as $[\~x\~y\~z]=\~\triangle(x y z)$.
The angle $\mangle\hinge{\~x}{\~y}{\~z}$ of the model triangle $[\~x\~y\~z]$ is called \emph{model angles} $[x y z]$ at $x$ and denoted as $\angk x y z$.

%???+PIC

Two geodesics $[xy]$ and $[xz]$ with common end $x$ is called \emph{hinge} and denoted as $\hinge{x}{y}{z}$.
The angle $\mangle\hinge{x}{y}{z}$ of the hinge is defined as the limit of model angles for small triangle sling along the sides of hinge to its vertex. Namely,
\[\mangle\hinge{x}{y}{z}=\lim_{\bar y,\bar z\to x}\set{\angk{x}{\bar y}{\bar z}}{ \bar y\in \left]xy\right], \bar z\in \left]xz\right]},\]
where $\left]xy\right]=[xy]\backslash\{x\}$.

\begin{thm}{Comparison property}
We say that hinge $\hinge x y z$ 
satisfies \emph{comparison} if the angle
$\mangle\hinge{x}{y}{z}$ is defined and 
\[\mangle \hinge{x}{y}{z} \ge \angk{x}{x}{y}{}{}.\]
If the comparison holds for all hinges in the surface $\Sigma$,
we say that $\Sigma$
has \emph{nonnegative curvature in the sence of Alexandrov}.
\end{thm}

The following two theorems play central role in the theory.
They admit generalizations to higher dimensions, but we need it only for surfaces.

\begin{thm}{Globalization theorem}
Assume comparison holds for all sufficiently small hinges in a surface $\Sigma$.
Then $\Sigma$ has non-negative curvature in the sense of Alexandrov.
\end{thm}

\begin{thm}{Comparison theorem}
Any surface of convex body in $\EE^3$,
if equipped with the induced intrinsic metric, 
is a surface with non-negative curvature in the sense of Alexandrov.
\end{thm}

In fact, the converse of this theorem also holds
if one extends the definition of \emph{surface} 
to the convex bodies which degenerate to 2-dimensional sets.
The surface of flat convex figure has to be defined as its doubling;
that is, two copies 
of the figure glued along the boundary --- you can imagine that you can walk on both sides of the figure but can not pass through it.

\begin{thm}{Theorem}
Any surface $\Sigma$ with non-negative curvature in the sense of Alexandrov which is homeomorphic to the sphere
is isometric to the surface of convex body $K$ in $\EE^3$,
which possibly degenerate to a flat figure.
\end{thm}

It turns out that $\Sigma$ defines $K$ up to congruence.
This a hard theorem was proved by Pogorelov.
We will use the following weaker statement
it was proved earlier by Alexandrov essentailly the same way as the Cauchy's rigidity theorem.

\begin{thm}{Rigidity theorem}
Polyhedrons in $\EE^3$ with isometric surfaces are congruent. 
\end{thm}

Let $\Sigma$ be a surface with non-negative curvature in the sense of Alexandrov.
Assume a triangle $[xyz]$ bounds in $\Sigma$ an open set $\Delta$ homeomorphic to a disc which is convex.
That is any minimizing geodesic with ends in $\Delta$
lie completely in $\Delta$.
In this case we define the \emph{curvature} of $\Delta$ as the angle excess of $[xyz]$;
that is,
\[\kappa(\Delta)=\angk x y z+\angk  y z x+\angk z x y-\pi.\]
It extends to non-negative measure, so called \index{curvature measure}\emph{curvature measure} in the unique way, defined on all Borel subsets in the interior of $\Sigma$.
The curvature of an interior point in $\Sigma$ can be thought as $\pi$ minus total angle around it;
a formal definition of \emph{total angle around point} is left to the reader.

For curvature measure, an analog of Gauss--Bonnet formula holds;
in particular, if $\Sigma$ is a surface with non-negative curvature in the sence of Alexandrov then
\begin{itemize}
\item If $\Sigma$ is homeomorphic to the sphere then 
\[\kappa(\Sigma)=4\cdot\pi\]
\item If a closed geodesic cuts from $\Sigma$ a disc $\Delta$ then 
\[\kappa(\Delta)=2\cdot\pi.\]
\end{itemize}


%???+PIC

We also will use the following comparison,
it follows easily from the comparison.

\begin{thm}{Theorem}
Assume $\Sigma$ is a surface with non-negative curvature in the sense of Alexandrov
and a triangle $[xyz]$ in $\Sigma$ bounds a open set $\Delta$ homeomorphic to a disc.
Then 
\[\area\Delta\ge \area\~\triangle(xyz).\]

\end{thm}

\section{Four singular points}

The following lemma is the key to the proof of Theorem~\ref{Long geodesic}.

\begin{thm}{Lemma} 
Assume that the surface $\Sigma$ of a convex body $B$ in $\mathbb{R}^3$
admits an arbitrary long simple closed geodesic.
Then the surface of $B$ contains $4$ singular points with curvature $\pi$ and the rest of it is flat.
\end{thm}

For the proof we will show that for any $\varepsilon$ is is possible to fount four non-intersecting open sets of diameter $\varepsilon$, such that each of them having curvature at least $\pi - \varepsilon$.
Going to the limit we will get the statement of the Lemma.

By cutting the surface $\Sigma$ along a sufficiently long closed simple geodesic,
we get two discs.
The key step is to show that each of these discs is long and thin.
Then we show that their ``corners'' form four required sets with big curvature.

\parit{Proof.}
Cut the surface $\Sigma$ along a sufficiently long closed simple geodesic,
we get two discs.
Choose one of the discs, say $D$;
equip it with the intrinsic metric further denoted by $|{*}-{*}|_D$.


Since $\Sigma$ has non-negative curvature in the sense of Alexandrov,
so is $D$.
Choose a pair of points $p,q\in\partial D$ which maximize the distance $|p-q|_D$.
Clearly,
\[|x-p|_D,|x-q|_D\le |p-q|_D\] 
for any other point $x\in\partial D$.
By comparison, 
\begin{equation}
	\label{eq:pxq>pi/3}
	\measuredangle[x\,^p_q]\ge \tfrac\pi3.\tag{${*}$}
\end{equation}




\begin{wrapfigure}{l}{28 mm}
\begin{lpic}[t(-0 mm),b(-0 mm),r(0 mm),l(0 mm)]{pics/long-geodesic-diam(1)}
\lbl[r]{0,8;$p$}
\lbl[l]{28,8;$q$}
\lbl[b]{10,16;$\gamma_1(t)$}
\lbl[l]{12,9,-60;$\ge\tfrac\pi3$}
\end{lpic}
\end{wrapfigure}

The points $p$ and $q$ divide $\partial D$ into two arcs,
say $\gamma_1$ and $\gamma_2$;
let us parametrize them by arclength from $p$ to $q$. 
Then by \eqref{eq:pxq>pi/3}
\begin{equation}
\tfrac{d}{dt}\left(|p-\gamma_i(t)|_D-|q-\gamma_i(t)|_D\right)
\ge
\tfrac12.
\tag{${*}{*}$}
\end{equation}

In particular
\[|p-q|_D\ge \tfrac18{\cdot}\length[\partial D].\]
That is, if the geodesic was long 
then $D$ has large diameter.

Fix small $\delta$ with respect to $\varepsilon$. 
(The value $\delta=\varepsilon\cdot\sin \varepsilon$ will do.
It means that any euclidean triangle with one side equal $\varepsilon$ and one side equals $\delta$ has an angle opposite to the side $\delta$ less than $\varepsilon$).

Let $x$ and $y$ be the first points along $\gamma_1$ and $\gamma_2$ at distance $\varepsilon$ from $p$.
If $|q-x|_D>|q-y|_D$, move the point $y$ along $\gamma_2$ toward $p$ until $|q-x|_D=|q-y|_D$; 
since $|p-q|_D$ is maximal, this distance will be achieved. 
In case $|q-x|_D<|q-y|_D$ do the same with the point $x$. 
Note that now $|q-x|_D=|q-y|_D\ge|p-q|_D-\varepsilon$.

By comparison 
\[\area \tilde \triangle qxy\le \area \triangle qxy\le \area \Sigma.\]
It follows that 
\begin{equation}
\label{eq:|x-y|}
\begin{aligned}
|x-y|_D&\le2\cdot\frac{ \area[\tilde\triangle(xyq)]}{|q-x|_D}
\le 
100\cdot\frac{ \area\Sigma}{\length[\partial D]}.
\end{aligned}
\tag{$***$}
\end{equation}
So we can choose $D$ having a large enough perimeter that $|x-y|_D <\delta$.

Analogously, we can prove that for any point $z$ in $D$ there is a point on $z'$ on $\gamma(t)$, that $|z-z'|_D$ less than $\delta$.
Indeed, one of the distance from $z$ to $p$ or $q$ is at least $|p-q|_D/2$, suppose it the point $q$.
Choose a point $z'$ on $\gamma(t)$ such that $|q-z'|_D=|q-z|_D$.
Considering the triangle $\tilde \triangle qzz'$ and inequality similar to \eqref{eq:|x-y|} we obtain $|z-z'|_D<\delta$.


Cut from $D$ along a minimizing geodesic $[xy]$
and consider part (a lune) $L_p$ with the point $p$ in it.
Note that the curvature of $L_p$ is $\alpha+\beta$, where $\alpha$ and $\beta$ the angles as on the diagram.
By comparison $\alpha\ge \tilde\measuredangle(x\,^p_y)$ 
and $\beta\ge \tilde\measuredangle(y\,^p_x)$.
Therefore curvature of $L_p$ is at least $\pi-\tilde\measuredangle(p\,^x_y)>\pi-\varepsilon$.
% In particular, if $|x-y|_D$ much less then $|p-x|_D+|p-y|_D$ then the curvature of $L_p$ is almost $\pi$.


Fix $\varepsilon>0$.
Assuming $\length[\partial D]$ is long enough, we can find a lune $L_p$ with perimeter at most $2\cdot\eps + \delta$,
such that curvature $L_p$ is at least $\pi-\eps$.

Using the same construction for $p$ and $q$ in the disc $D$,
and for the other disc,
we get four lunes in $\Sigma$ each of diameter at most $\eps$ and each with curvature at least $\pi-\eps$.

\begin{center}
\begin{lpic}[t(3 mm),b(3 mm),r(0 mm),l(0 mm)]{pics/long-geodesic-D(1)}
\lbl{46,6;$D$}
\lbl{6.5,5.7;{\small $L_p$}}
\lbl[r]{0,6;$p$}
\lbl[l]{92.5,6;$q$}
\lbl[b]{19,11.8;$x$}
\lbl[t]{12.5,.5;$y$}
\lbl[b]{46,12;$\gamma_1$}
\lbl[t]{46,-.5;$\gamma_2$}
\lbl[tr]{16.5,9;{\small $\alpha$}}
\lbl[br]{12.5,4.5;{\small $\beta$}}
\end{lpic}
\end{center}

By Gauss--Bonnet formula, the total curvature of $\Sigma$ is $4\cdot\pi$.
Since $\eps>0$ is arbitrary, we get that there are four singular points in $\Sigma$, each with curvature $\pi$
and the remaining part of $\Sigma$ is flat.

\section{Isosceles tetrahedron}

\begin{thm}{Lemma} 
Assume that the closed convex surface $\Sigma$ in $\RR^3$
has 4 singular points with curvature $\pi$.
Then it bounds an isosceles tetrahedron.
\end{thm}

{

\begin{wrapfigure}{o}{21 mm}
\begin{lpic}[t(-4 mm),b(-3 mm),r(0 mm),l(0 mm)]{pics/akopyan(1)}
\end{lpic}
\end{wrapfigure}

\parit{Proof.}
Denote the singular points by $p$, $x$, $y$ and $z$.

Let us cut $\Sigma$ along three geodesics $[px]$, $[py]$ and $[pz]$.
Develop the obtained flat surface on the plane.
Note that as the result we obtain a triangle $\triangle$; 
the points $x$, $y$ and $z$ correspond to the midpoints of the sides of $\triangle$
and the point $p$ correspond to the vertices of $\triangle$.

}

It follows that $\Sigma$ is isometric to the surface of isosceles tetrahedron.
The statement now follows from the Pogorelov's uniqueness theorem \cite{pogorelov},
which states that if two closed convex surfaces have isometric intrinsic metrics then they bound congruent bodies.

Alternatively one could apply the following exercise and the uniqueness theorem for the surfaces polyhedra, which has much simple proof, see \cite{alexandrov1950}.
\qeds

\begin{thm}{Exercise}
Assume that a convex body $K$ is bounded by a closed flat surface with finite number of singular points.
Show that $K$ is a polyhedron.
\end{thm}

\begin{thebibliography}{52}
\bibitem{alexandrov1941} 
Alexandroff, A. 
The inner geometry of an arbitrary convex surface. C. R. (Doklady) 
Acad. Sci. URSS (N.S.) 
32, 
(1941). 
467--470

\bibitem{alexandrov1950} 
Alexandrov, A. D.,
\emph{Convex polyhedra.}
Springer Monographs in Mathematics. Springer-Verlag, Berlin, 2005.

\bibitem{pogorelov} 
\begin{otherlanguage}{russian}
Погорелов, А. В.,
\emph{Однозначаная определённость общих выпукых поверхностей}
Киев, 1952.
\end{otherlanguage}



\bibitem{protasov} Protasov, V. Yu.,
\emph{On the number of closed geodesics on a polyhedron.}  Russian Math. Surveys 63 (2008), no. 5, 978--980
\end{thebibliography}


\end{document}
