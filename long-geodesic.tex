\documentclass[oneside,a4paper]{article}
\usepackage{long-geodesic}

\usepackage{lineno}
\linenumbers

\begin{document}

\title{Long geodesics on convex surfaces}
\author{A. Akopyan and A. Petrunin}
\date{}
\maketitle

\begin{abstract}
We review the theory of intrinsic geometry of convex surfaces in the Euclidean space and prove the following theorem: 
if the surface of convex body $K$ contains arbitrary long closed simple geodesic, then $K$ is an isosceles tetrahedron.
\end{abstract}

\section{Introduction}

The goal of this note is to introduce the reader to the theory of intrinsic geometry of convex surfaces.

After reviewing the theory of convex surfaces and prove the following theorem which generalize the main theorem proved by Vladimir Protasov in \cite{protasov}.
This theorem is used as an illustration.

\begin{wrapfigure}{r}{21 mm}
\begin{lpic}[t(-4 mm),b(-0 mm),r(0 mm),l(0 mm)]{pics/long-geodesic(1)}
\end{lpic}
\end{wrapfigure}

Recall that a curve in a convex surface is called geodesic if every sufficiently short arc of the curve is length minimizing
and we say that it is simple if it has no self intersections.

A tetrahedron with equal opposite edges is called \emph{isosceles}.

\begin{thm}{Theorem}\label{Long geodesic}
Assume that the surface $\Sigma$ of a convex body $B$ in $\RR^3$
admits an arbitrary long simple closed geodesic.
Then $B$ is an isosceles tetrahedron.
\end{thm}
 

\section{The theory}

A very short but comprehensive introduction to the subject was written by Alexander Alexandrov in \cite{alexandrov1941}.

TODO: Tangent cone, direction of curve.
Liberman's Lemma, direction of geodesic.
Angle between geodesics.
Model triangle/model angle.
First variation formula.


On a convex surface (not necessary smooth) one could define 
so
called \index{curvature measure}\emph{curvature measure} which we denote further by $\kappa$.
It is the (necessary unique) non-negative measure such that for any triangle $\triangle$, we have
\[\kappa(\triangle)=\alpha+\beta+\gamma-\pi,\] 
where $\alpha$, $\beta$ and $\gamma$ are the angles of $\triangle$, measured in the intrinsic metric of the surface.

For curvature measure, an analog of Gauss--Bonnet formula holds;
in particular
\[\kappa(\Sigma)=4\cdot\pi\]
for any closed convex surface $\Sigma$ in $\RR^3$

%???+PIC

Further, given a triangle $\triangle$ in a metric space,
its model triangle $\tilde \triangle$ is defined as a triangle in the plane with the same side lengths.
The angles $\tilde\alpha$, $\tilde\beta$ and $\tilde\gamma$ of the model triangle are called \index{model angle}\emph{model angle}
of triangle.
The comparison theorem states that for any triangle in a surface with non-negative curvature measure its model angles do not exceed the actual angles; that is,
\[\tilde\alpha\le \alpha,\quad
\tilde\beta\le\beta,\quad
\tilde\gamma\le\gamma.\]
The same holds for the area; that is,
\[\area\tilde\triangle\le \area\triangle.\]

\section{Four singular points}

The following lemma is the key to the proof.

\begin{thm}{Lemma} 
Assume that the surface $\Sigma$ of a convex body $B$ in $\RR^3$
admits an arbitrary long simple closed geodesic.
Then the surface of $B$ contains 4 singular points with curvature $\pi$ and the rest of it is flat.
\end{thm}


By cutting the surface $\Sigma$ along a sufficiently long closed simple geodesic,
we get two discs.
The key step is to show that each of these discs 
is long and thin.

\medskip

Choose one of the discs, say $D$;
equip it with the intrinsic metric further denoted by $|{*}-{*}|_D$.


Since $\Sigma$ has non-negative curvature in the sense of Alexandrov,
so is $D$.
Choose a pair of points $p,q\in\partial D$ which maximize the distance $|p-q|_D$.
Clearly,
\[|x-p|_D,|x-q|_D\le |p-q|_D\] 
for any other point $x\in\partial D$.
By comparison, 
\[\measuredangle[x\,^p_q]\ge \tfrac\pi3.\leqno({*})\]



\begin{wrapfigure}{r}{28 mm}
\begin{lpic}[t(-0 mm),b(-0 mm),r(0 mm),l(0 mm)]{pics/long-geodesic-diam(1)}
\lbl[r]{0,8;$p$}
\lbl[l]{28,8;$q$}
\lbl[b]{10,16;$\gamma_1(t)$}
\lbl[l]{12,9,-60;$\ge\tfrac\pi3$}
\end{lpic}
\end{wrapfigure}

The points $p$ and $q$ divide $\partial D$ into two arcs,
say $\gamma_1$ and $\gamma_2$;
let us parametrize them by arclength from $p$ to $q$. 
Then by $({*})$
\[\tfrac{d}{dt}\left(|x-\gamma_i(t)|_D-|x-\gamma_i(t)|_D\right)
\ge
\tfrac12.\]
In particular
\[|p-q|_D\ge \tfrac18{\cdot}\length[\partial D].\]
That is, if the geodesic was long 
then $D$ has large diameter.

Choose two points $x\in \gamma_1$ and $y\in\gamma_2$ sufficiently close to $p$ such that $|x-q|_D=|y-q|_D$.
By comparison 
\[\area \tilde \triangle qxy\le \area \triangle qxy\le \area \Sigma.\]
It follows that 
\[\begin{aligned}|x-y|_D&\le2\cdot\frac{ \area[\tilde\triangle(xyq)]}{|q-x|_D}
\le 
\\
&\le 
100\cdot\frac{ \area\Sigma}{\length[\partial D]}.
\end{aligned}
\leqno({*}{*})\]

Cut from $D$ along a minimizing geodesic $[xy]$
and consider part (a loon) $L_p$ with the point $p$ in it.
Note that the curvature of $L_p$ is $\alpha+\beta$, where $\alpha$ and $\beta$ the angles as on the diagram.
By comparison $\alpha\ge \tilde\measuredangle(x\,^p_y)$ 
and $\beta\ge \tilde\measuredangle(y\,^p_x)$.
Therefore curvature of $L_p$ is at least $\pi-\tilde\measuredangle(p\,^x_y)$.
In particular, if $|x-y|_D$ much less then $|p-x|_D+|p-y|_D$ then the curvature of $L_p$ is almost $\pi$.

Fix $\eps>0$.
If $\length[\partial D]$ is long enough,
by $({*})$, 
we can find a loon $L_p$ with diameter at most $\eps$,
such that curvature $L_p$ is at least $\pi-\eps$.

Using the same construction for $p$ and $q$ in the disc $D$,
and for the other disc,
we get four loons in $\Sigma$ each of diameter at most $\eps$ and each with curvature at least $\pi-\eps$.

\begin{center}
\begin{lpic}[t(-0 mm),b(-0 mm),r(0 mm),l(0 mm)]{pics/long-geodesic-D(1)}
\lbl{46,6;$D$}
\lbl{6,6;$L_p$}
\lbl[r]{0,6;$p$}
\lbl[l]{92.5,6;$q$}
\lbl[b]{19,11.5;$x$}
\lbl[t]{12.5,.5;$y$}
\lbl[b]{46,12;$\gamma_1$}
\lbl[t]{46,-.5;$\gamma_2$}
\lbl[tr]{16.5,9;{\small $\alpha$}}
\lbl[br]{12.5,4.5;{\small $\beta$}}
\end{lpic}
\end{center}

By Gauss--Bonnet formula, the total curvature of $\Sigma$ is $4\cdot\pi$.
Since $\eps>0$ is arbitrary, we get that there are four singular points in $\Sigma$, each with curvature $\pi$
and the remaining part of $\Sigma$ is flat.

\section{Isosceles tetrahedron}

\begin{thm}{Lemma} 
Assume that the closed convex surface $\Sigma$ in $\RR^3$
has 4 singular points with curvature $\pi$.
Then it bounds an isosceles tetrahedron.
\end{thm}

{

\begin{wrapfigure}{o}{21 mm}
\begin{lpic}[t(-4 mm),b(-3 mm),r(0 mm),l(0 mm)]{pics/akopyan(1)}
\end{lpic}
\end{wrapfigure}

\parit{Proof.}
Denote the singular points by $p$, $x$, $y$ and $z$.

Let us cut $\Sigma$ along three geodesics $[px]$, $[py]$ and $[pz]$.
Develop the obtained flat surface on the plane.
Note that as the result we obtain a triangle $\triangle$; 
the points $x$, $y$ and $z$ correspond to the midpoints of the sides of $\triangle$
and the point $p$ correspond to the vertices of $\triangle$.

}

It follows that $\Sigma$ is isometric to the surface of isosceles tetrahedron.
The statement now follows from the uniqueness theorem \cite{???},
which states that if two closed convex surfaces have isometric intrinsic metrics then they bound congruent bodies.
Alternatively one could apply the following exercise and the uniqueness theorem for the surfaces polyhedra, which has much simple proof, see \cite{alexandrov1950}.
\qeds

\begin{thm}{Exercise}
Assume that a convex body $K$ ids bounded by a closed flat surface with finite number of singular points.
Show that $K$ is a polyhedron.
\end{thm}

\begin{thebibliography}{52}
\bibitem{alexandrov1941} 
Alexandroff, A. 
The inner geometry of an arbitrary convex surface. C. R. (Doklady) 
Acad. Sci. URSS (N.S.) 
32, 
(1941). 
467--470

\bibitem{alexandrov1950} 
Alexandrov, A. D.
\emph{Convex polyhedra.}
Springer Monographs in Mathematics. Springer-Verlag, Berlin, 2005.

\bibitem{protasov} Protasov, V. Yu. 
On the number of closed geodesics on a polyhedron.  Russian Math. Surveys 63 (2008), no. 5, 978--980
\end{thebibliography}


\end{document}
