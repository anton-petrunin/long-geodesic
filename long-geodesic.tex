\documentclass[oneside,a4paper, 12pt]{article}
\usepackage{anysize}
%\marginsize{left}{right}{top}{bottom}
\marginsize{2cm}{2cm}{1cm}{2cm}
\usepackage{long-geodesic}
\hypersetup{pdftitle={Long geodesics on convex surfaces},
pdfauthor={Arseniy Akopyan and Anton Petrunin}}

%\usepackage{lineno}\linenumbers

\begin{document}

\title{Long geodesics on convex surfaces}
\author{Arseniy Akopyan and Anton Petrunin}
\date{}
\maketitle

\begin{abstract}
We review the theory of intrinsic geometry of convex surfaces in the Euclidean space and prove the following theorem: 
if the surface of convex body $K$ contains arbitrary long closed simple geodesic, then $K$ is an isosceles tetrahedron.
\end{abstract}

\section{Introduction}

The goal of this note is to introduce the reader to the theory of intrinsic geometry of convex surfaces.
We illustrate the power of the tools by proving the theorem below.

\begin{wrapfigure}{r}{21 mm}
\begin{lpic}[t(-4 mm),b(-0 mm),r(0 mm),l(0 mm)]{pics/long-geodesic(1)}
\end{lpic}
\end{wrapfigure}

Recall that a curve in a convex surface is called \emph{geodesic} if every sufficiently short arc of the curve is length minimizing;
if it has no self intersections, we call it \emph{simple geodesic}.

A tetrahedron with equal opposite edges is called \emph{isosceles}.

\begin{thm}{Theorem}
	\label{Long geodesic}
Assume that the surface $\Sigma$ of a convex body $B$ in the Euclidean space $\EE^3$
admits an arbitrary long simple closed geodesic.
Then $B$ is an isosceles tetrahedron.
\end{thm}

In \cite{protasov2008onthenumber}, Vladimir Protasov proved this result for the surfaces of convex polyhedrons.
 

\section{An overview of the theory}
\subsection*{Angles}
The intrinsic metric on convex surfaces has certain comparison property, which we about to discuss.
A short introduction to the subject was written by Alexander Alexandrov in \cite{alexandroff1941theinner}. 
Later he wrote a comprehensive book on the subject \cite{aleksandrov1948vnutrennnyaya, alexandrow1955dieinnere} which we recommend to anyone who can read Russian or German,
otherwise read the book of Herbert Busemann \cite{busemann1958convex}.

By \emph{surface} we will understand a compact $2$-dimensional manifold equipped with \emph{geodesic} metric, 
possibly with non-empty boundary.
A metric on the surface $\Sigma$ is called geodesic if any two points $p,q\in \Sigma$ can be joined by a curve with length $|p-q|_\Sigma$;
we denote by $|p-q|_\Sigma$ the distance from $p$ to $q$ in $\Sigma$.
This curve will be denoted as $[pq]$ and called a \emph{minimizing geodesic} from $p$ to $q$.

For a triple of points $x,y,z\in\Sigma$, together with a choice of a three minimizing geodesics $[x y], [y z], [z x]$ will be called a \index{triangle}\emph{triangle} 
and denoted as 
$[x y z]$.
Many different triangles with vertices $x$, $y$ and $z$ may exist, 
any of which can be denoted by $[x y z]$.

A triangle $[\~x\~y\~z]$ in the plane $\EE^2$
with the same side lengths as $[x y z]$ 
is called \emph{model triangle} of $[x y z]$;
this relation will be written as $[\~x\~y\~z]=\~\triangle(x y z)$.
The angle $\mangle\hinge{\~x}{\~y}{\~z}$ of the model triangle $[\~x\~y\~z]$ is called \emph{model angles} $[x y z]$ at $x$ and denoted as $\angk x y z$.

\begin{wrapfigure}{r}{19 mm}
\begin{lpic}[t(-5 mm),b(0 mm),r(0 mm),l(0 mm)]{pics/hinge(1)}
\lbl[t]{17,2;$x$}
\lbl[t]{8,0;$\bar x$}
\lbl[r]{5,17;$y$}
\lbl[r]{1.5,8;$\bar y$}
\lbl[rt]{1,0;$p$}
\end{lpic}
\end{wrapfigure}



Two geodesics $[xy]$ and $[xz]$ with common end $x$ is called \emph{hinge} and denoted as $\hinge{x}{y}{z}$.
The angle $\mangle\hinge{x}{y}{z}$ of the hinge is defined as the limit of model angles for triangles sling along the sides of hinge to its vertex. 
That is,
\[\mangle\hinge{x}{y}{z}=\lim_{\bar y,\bar z\to x}\set{\angk{x}{\bar y}{\bar z}}{ \bar y\in \left]xy\right], \bar z\in \left]xz\right]},\]
where $\left]xy\right]=[xy]\backslash\{x\}$.

Note that if $p\in \left]xy\right[=[xy]\backslash\{x,y\}$ then $\mangle\hinge{p}{x}{y}=\pi$.

In general, the angle of hinge maybe undefined, but as you will see it will be defined all the time we need it.

\subsection*{Comparison}

\begin{thm}{Comparison property}\label{Comparison property}
We say that hinge $\hinge x y z$ 
satisfies the \emph{comparison property} if the angle
$\mangle\hinge{x}{y}{z}$ is defined and 
\[\mangle \hinge{x}{y}{z} \ge \angk{x}{y}{z}{}{}.\]
\end{thm}

\begin{wrapfigure}{r}{27 mm}
\begin{lpic}[t(-0 mm),b(0 mm),r(0 mm),l(0 mm)]{pics/suplimentary(1)}
\lbl[t]{24,2.6;$x$}
\lbl[t]{1,13.5;$y$}
\lbl[t]{5,-.5;$z$}
\lbl[bl]{15,13;$p$}
\end{lpic}
\end{wrapfigure}

Two hinges $\hinge{p}{x}{z}$ and $\mangle\hinge{p}{y}{z}$ will be called \emph{supplementary},
if they share side $[pz]$ and $p\in \left]xy\right[$.

\begin{thm}{Supplementary property}\label{Supplementary property}
We say that \emph{suplementary property} holds for two supplementary hinges $\hinge{p}{x}{z}$ and $\mangle\hinge{p}{y}{z}$ if the angles $\mangle\hinge{p}{x}{z}$ and $\mangle\hinge{p}{y}{z}$ are defined and
\[\mangle\hinge{p}{x}{z}+\mangle\hinge{p}{y}{z}=\pi.\]

\end{thm}


If in the surface $\Sigma$, the angles of all hinges are defined and satisfy the comparison and supplementary properties, then we say that $\Sigma$ has \emph{non-negative curvature in the sense of Alexandrov}.%
\footnote{It is not known if well defined angles together with comparison property imply supplementary property.}


The geometry of such surfaces is very particular. 
For example, by the comparison property, if a hinge has vanishing angle then one of its sides lies in the other.
In particular,  geodesics in $\Sigma$ can not bifurcate.

Further, assume $\gamma$ is a geodesic in $\Sigma$, 
and $p$ is arbitrary point on $\Sigma$.
Note that the function $f(t)=|p-\gamma(t)|_\Sigma$ is  1-Lipschitz. 
In particular $f$ is differentiable almost everywhere.

\begin{wrapfigure}{r}{25 mm}
\begin{lpic}[t(-0 mm),b(0 mm),r(0 mm),l(0 mm)]{pics/first-variation(1)}
\lbl[t]{5,0;$p$}
\lbl[bl]{15,13;$\gamma(t)$}
\lbl[rt]{10,12;$\phi_{+}$}
\lbl[lt]{15,8;$\phi_{-}$}
\end{lpic}
\end{wrapfigure}

Assume $\phi_\pm(t)$ are the angles between the positive and negative directions of $\gamma$ and a geodesic $[\gamma(t)p]$.
Applying the definition of angle and triangle inequality,
one gets the following
\[\pm f'(t)\le -\cos[\phi_\pm(t)]\]
at any $t$ such that the derivative $f'(t)$ is defined.

\begin{thm}{Exercise}
Prove the last statement.
\end{thm}

%
Applying the supplementary property we get that $\phi_-+\phi_+=\pi$, hence $\cos\phi_+ +\cos\phi_-=0$.
Therefore the two inequalities above imply the identity
\begin{equation}
	\label{eq:first variation}
f'(t)=-\cos[\phi_+(t)]
	\tag{${*}$}
\end{equation}
for any $t$, where the derivative $f'(t)$ is defined.
This statement is regarded as \emph{first variation formula}.

Further, for any surface $\Sigma$ with non-negative curvature in the sense of Alexandrov,
the Kirszbraun extension theorem holds.
That is, any distance non-expanding map from a subset of surface to the Euclidean plane can be extended to a distance non-expanding map defined on whole $\Sigma$;
see \cite{lang1997kirszbraun,alexander2011alexandrov}.
(In fact, the statement in Kiszbraun could be used as the definition of spaces with non-negative curvature in the sense of Alexandrov.)
Applying Kiszbraun theorem for three-point sets,
one gets the following area comparison which will be important to us.

\begin{thm}{Area comparison}\label{Area comparison}
Assume $\Sigma$ is a surface with non-negative curvature in the sense of Alexandrov
and a triangle $[xyz]$ in $\Sigma$ bounds a open set $\Delta$ homeomorphic to a disc.
Then 
\[\area\Delta\ge \area\~\triangle(xyz).\]

Another proof of this theorem by dissection on small triangles can be found in \cite[Chapter X \S 1]{ aleksandrov1948vnutrennnyaya, alexandrow1955dieinnere}.

\end{thm}

\subsection*{Convex surfaces}
The following two theorems play central role in the theory.
They admit generalizations to higher dimensions, but we need it only for surfaces.

\begin{thm}{Globalization theorem}\label{Globalization theorem}
Assume that any point the surface $\Sigma$ admits a neighborhood $U$ such that the angles of all hinges in $U$ are defined and satisfy 
the comparison and supplementary properties.
Then these properties hold for all hinges;
that is $\Sigma$ has non-negative curvature in the sense of Alexandrov.
\end{thm}

An amusing proof of the theorem above is given by Urs Lang and Viktor Schroeder in \cite{lang2012toponogov}.

Recall that the intrinsic distance between points $x$ and $y$ on a surface $\Sigma$ in $\EE^3$, is defined as the least upper bound for the lengths of curves connecting $x$ to $y$ in $\Sigma$.

\begin{thm}{Comparison theorem}\label{Comparison theorem}
Any surface of convex body in $\EE^3$,
if equipped with the induced intrinsic metric, 
is a sphere with a non-negatively curved metric in the sense of Alexandrov.
\end{thm}

In fact, as it was prove by Alexandrov in \cite[Chapter III \S 3]{aleksandrov1948vnutrennnyaya, alexandrow1955dieinnere},
the converse of this theorem also holds
if one considers convex plane figure as a convex body.
The surface of flat convex figure has to be defined as its doubling;
that is, two copies 
of the figure glued along the boundary --- that is like a surface if you can walk on both sides of the figure but can not pass through it.

\begin{thm}{Theorem}
Any surface $\Sigma$ with non-negative curvature in the sense of Alexandrov which is homeomorphic to the sphere
is isometric to the surface of a convex body in $\EE^3$,
which possibly degenerate to a flat figure.
\end{thm}

It turns out that $\Sigma$ defines the convex body up to congruence.
This hard theorem for polyhedral metrics was proved by Alexandrov \cite[Chapter VI]{aleksandrov1948vnutrennnyaya, alexandrow1955dieinnere} for polyhedral metrics and by Pogorelov in full generality \cite{pogorelov1952odnoznacnaya}.
We will use the following weaker statement, which goes essentially the same way as the Cauchy's rigidity theorem.

\begin{thm}{Rigidity theorem}\label{Rigidity theorem}
Convex polyhedrons in $\EE^3$ with isometric surfaces are congruent. 
\end{thm}

\subsection*{Curvature measure}

Let $\Sigma$ be a surface with non-negative curvature in the sense of Alexandrov.
Assume a triangle $[xyz]$ bounds in $\Sigma$ an open set $\Delta$ homeomorphic to a disc which is convex.
That is any minimizing geodesic with ends in $\Delta$
lie completely in $\Delta$.
In this case we define the \emph{curvature} of $\Delta$ as the angle excess of $[xyz]$;
that is,
\[\kappa(\Delta)=\mangle \hinge x y z+\mangle \hinge  y z x+\mangle \hinge z x y-\pi.\]
It extends to non-negative measure, so called the \index{curvature measure}\emph{curvature measure} in the unique way, defined on all Borel subsets in the interior of $\Sigma$.
%(If $\Sigma$ is a convex surface in $\mathbb{E}^3$, then the curvature measure of a set in $\Sigma$ can be thought as the Lebesgues measure of its spherical image of $U$ under the the map .) Правильно такое сказать тяжело --- поверхность не гладкая, сферическое отображение неоднозначное ...

\begin{wrapfigure}{r}{23 mm}
\begin{lpic}[t(-7 mm),b(-0 mm),r(0 mm),l(0 mm)]{pics/geodesic(1)}
%\lbl[r]{0,8;$p$}
\end{lpic}
\end{wrapfigure}

It turns out that in $\Sigma$, any geodesic without its end-points has vanishing curvature.
This can be proved by covering interior of geodesic by two thin triangles with small excess as shown on the picture.
(This requires some work, but simple.)

\begin{wrapfigure}{l}{20 mm}
\begin{lpic}[t(-0 mm),b(-0 mm),r(0 mm),l(0 mm)]{pics/mercedes(1)}
%\lbl[r]{0,8;$p$}
\end{lpic}
\end{wrapfigure}

Further the curvature of an interior point in $\Sigma$ can be thought as $\pi$ minus total angle around it.
The later can be seen from the picture ---
to find the curvature of the central vertex one has to subtract excesses of three small triangles from the excess of the big one.
(The existence of such configuration also requires some work.)

For curvature measure, an analog of Gauss--Bonnet formula holds;
in particular, if $\Sigma$ is a surface with non-negative curvature in the sence of Alexandrov then
\begin{itemize}
\item If $\Sigma$ is homeomorphic to the sphere then 
\[\kappa(\Sigma)=4\cdot\pi\]
\item If a closed geodesic cuts from $\Sigma$ a disc $\Delta$ then 
\[\kappa(\Delta)=2\cdot\pi.\]
\end{itemize}


\section{Four singular points}

The following lemma is the key to the proof of Theorem~\ref{Long geodesic}.

A point in a surface is called \emph{flat} if it admits a flat neighborhood;
that is a neighborhood isometric to an open subset of the plane.

\begin{thm}{Lemma} 
	\label{lem:4 singular points}
Assume that the surface $\Sigma$ of a convex body $B$ in $\mathbb{R}^3$
admits an arbitrary long simple closed geodesic.
Then the surface of $B$ contains $4$ singular points with curvature $\pi$ and the rest of it is flat.
\end{thm}

For the proof we will show that for any $\varepsilon>0$, 
it is possible to found four non-intersecting open sets of diameter $\varepsilon$, such that each of them having curvature at least $\pi - \varepsilon$.
Going to the limit we will get the statement of the Lemma.

By cutting the surface $\Sigma$ along a sufficiently long closed simple geodesic,
we get two discs.
The key step is to show that each of these discs is long and thin.
Then we show that their ``corners'' form four required sets with big curvature.

To be more precise the founded four sets will have small perimeter.
The following claim implies that these conditions are equivalent.
In fact, this claim holds for any metric on the sphere, not necessary closed convex surface.

\begin{thm}{Claim}\label{Lemma:diameter-perimeter}
For any $\varepsilon>0$ there is a $\delta>0$, that any simple curve in $\Sigma$ of length smaller than $\delta$ bounds a region of diameter at most $\varepsilon$.
\end{thm}

\parit{Proof.}
Arguing by contradiction, suppose there is a sequence of curves $\gamma_n$ which cuts $\Sigma$ into open regions $A_n$ and $B_n$ of diameter at least $\varepsilon$ each and such that $\length\gamma_n\to 0$ as $n\to\infty$. 
By compactness of $\Sigma$,
we can pass to a subsequence of $\gamma_n$ which converges to a point, denote it by $p$. 

Note that for large $n$, each region contains a disk with centers $a_n$ and $b_n$ and radius $\tfrac\varepsilon3$. 
Indeed, if $n$ is large, then $\gamma_n$ lies in $\tfrac\eps3$ neighborhood of $p$.
Since the diameter of regions is at least $\eps$, points in the regions maximizing the distance to $p$ will do the trick.

Pass to a subsequence of $\gamma_n$ so that $a_n$ and $b_n$ converge, denote by $a$ and $b$ their limits.
Note that for large $n$ the domains $A_n$ and $B_n$ contain the disks of radius $\tfrac\varepsilon4$ centered at $a$ and $b$ correspondingly.
In particular, for all large $n$, any path from $a$ to $b$ have to pass through $\gamma_n$.
Since $\gamma_n$ converges to $p$, any path from $a$ to $b$ pass through $p$.
The later does not hold since $\Sigma$ is homeomorphic to the sphere, a contradiction.	
\qeds


\parit{Proof of Lemma~\ref{lem:4 singular points}}
Cut the surface $\Sigma$ along a sufficiently long closed simple geodesic,
we get two discs.
Choose one of the discs, say $D$;
equip it with the intrinsic metric further denoted by $|{*}-{*}|_D$.


According to the comparison theorem (\ref{Comparison theorem}) the surface $\Sigma$ has non-negative curvature in the sense of Alexandrov.
By globalization theorem (\ref{Globalization theorem})
the same holds for $D$.

Choose a pair of points $p,q\in\partial D$ which maximize the distance $|p-q|_D$.
Clearly,
\[|x-p|_D,|x-q|_D\le |p-q|_D\] 
for any other point $x\in\partial D$.
By comparison property (\ref{Comparison property}) 
\begin{equation}
	\label{eq:pxq>pi/3}
	\measuredangle[x\,^p_q]\ge \tfrac\pi3.
	\tag{${**}$}
\end{equation}

The points $p$ and $q$ divide $\partial D$ into two arcs,
say $\gamma_1$ and $\gamma_2$;
let us parametrize them by arclength from $p$ to $q$. 

\begin{wrapfigure}[6]{r}{31mm}
\begin{lpic}[t(2 mm),b(-0 mm),r(0 mm),l(1 mm)]{pics/long-geodesic-diam(1)}
\lbl[r]{0,8;$p$}
\lbl[l]{28,8;$q$}
\lbl[b]{10,16;$\gamma_1(t)$}
\lbl[l]{12,9,-60;$\ge\tfrac\pi3$}
\end{lpic}
\end{wrapfigure}

By the first variation formula \eqref{eq:first variation}, for almost all $t$ we have
\[\tfrac{d}{dt}|p-\gamma_i(t)|_D=-\cos \phi,\] 
where $\phi$ is the angle between the direction of $\gamma_i$ at $x=\gamma_i(t)$ and the geodesic $[xp]$.
The same way 
\[\tfrac{d}{dt}|q-\gamma_i(t)|_D=-\cos \psi,\] 
where $\psi$ is the angle between the direction of $\gamma_i$ at $x\z=\gamma_i(t)$ and the geodesic $[xq]$.
By \eqref{eq:pxq>pi/3} and the supplementary property (\ref{Supplementary property}) we have $\phi-\psi\ge \tfrac\pi3$.
Therefore 
\begin{equation*}
\tfrac{d}{dt}\left(|p-\gamma_i(t)|_D-|q-\gamma_i(t)|_D\right)
= \cos \psi-\cos\phi
\ge
\tfrac12.
% \tag{${*}{*}$}
\end{equation*}

In particular
\[|p-q|_D\ge \tfrac18{\cdot}\length[\partial D].\]
That is, if the geodesic was long 
then $D$ has large diameter.

Fix a positive $\delta$ which is small with respect to $\varepsilon$. 
(The value $\delta=\varepsilon\cdot\sin \varepsilon$ will do.
It means that in any euclidean triangle with one side $\varepsilon$ and another $\delta$ has the angle opposite $\delta$ less than $\varepsilon$).

Let $x$ and $y$ be the first points along $\gamma_1$ and $\gamma_2$ at distance $\varepsilon$ from $p$.
If $|q-x|_D>|q-y|_D$, move the point $y$ along $\gamma_2$ toward $p$ until $|q-x|_D=|q-y|_D$; 
since $|p-q|_D$ is maximal, this distance will be achieved. 
In case $|q-x|_D<|q-y|_D$ do the same with the point $x$. 
Now we have that
$$|q-x|_D=|q-y|_D\ge|p-q|_D-\varepsilon.$$

By the area comparison (\ref{Area comparison})
\[\area \tilde \triangle qxy\le \area \triangle qxy\le \area \Sigma.\]
It follows that 
\begin{equation*}
\label{eq:|x-y|}
\begin{aligned}
|x-y|_D&\le2\cdot\frac{ \area[\tilde\triangle(xyq)]}{|q-x|_D}
\le 
100\cdot\frac{ \area\Sigma}{\length[\partial D]}.
\end{aligned}
% \tag{${*}{*}{*}$}
\end{equation*}
Therefore, if $D$ has large perimeter then $|x-y|_D <\delta$.

Cut $D$ by $[xy]$
and consider the part (a lune) $L_p$ with the point $p$ in it.
Note that the curvature of $L_p$ is $\alpha+\beta$, where $\alpha$ and $\beta$ the angles as on the diagram.
By comparison $\alpha\ge \tilde\measuredangle(x\,^p_y)$ 
and $\beta\ge \tilde\measuredangle(y\,^p_x)$.
Therefore curvature of $L_p$ is at least $\pi-\tilde\measuredangle(p\,^x_y)>\pi-\varepsilon$.

\begin{center}
\begin{lpic}[t(3 mm),b(3 mm),r(0 mm),l(0 mm)]{pics/long-geodesic-D(1)}
\lbl{46,6;$D$}
\lbl{6.5,5.7;{\small $L_p$}}
\lbl[r]{0,6;$p$}
\lbl[l]{92.5,6;$q$}
\lbl[b]{19,11.8;$x$}
\lbl[t]{12.5,.5;$y$}
\lbl[b]{46,12;$\gamma_1$}
\lbl[t]{46,-.5;$\gamma_2$}
\lbl[tr]{16.5,9;{\small $\alpha$}}
\lbl[br]{12.5,4.5;{\small $\beta$}}
\end{lpic}
\end{center}


Fix $\varepsilon>0$.
Assuming $\length[\partial D]$ is long enough, we can find a lune $L_p$ with perimeter at most $2\cdot\eps + \delta$,
such that curvature $L_p$ is at least $\pi-\eps$.
If $\eps$ is small, by Lemma \ref{Lemma:diameter-perimeter}, $L_p$ has small diameter, say $\eps'=\eps'(\eps)$.

Using the same construction for $p$ and $q$ in the disc $D$,
and for the other disc,
we get four lunes in $\Sigma$, 
each of diameter at most $\eps'$, 
and each with curvature at least $\pi-\eps$.
By Gauss--Bonnet formula, the remaining curvature is at most $4\cdot\eps$.

Since $\eps>0$ is arbitrary, it follows that support of curvature measure can be covered by four sets with arbitrary small diameter.
The support of the measure is a $4$-point set;
clearly each of these points has curvature $\pi$.
The remaining part of $\Sigma$ has vanishing curvature and therefore flat.
\qeds


%Let us go to the limit and show that there are four distinct singular points of curvature $\pi/2$ each.
%%Choose a sequence of quadruples $\{L_n^1, L_n^2, L_n^3, L_n^4\}$ of loons constructed by above descirbed way, which diameters tend to zero, and points $a_n^i\in L_n^i$. 
%By compactness of $\Sigma$ there is a subsequence of quadruples that $a^i_n \to a^i$, for $i=1,..,4$.

%First, note that $a^1$ has curvature at least $\pi/2$. Indeed, any open neighborhood contains $L_n^1$ for sufficiently big $n$, therefore curvature of any open sets containing $a^1$ is at least $\pi/2$.
%%On the other hand for sufficiently big $n$ there is a neighborhood of $a^1$ not containing $L_n^i$, $i=2,3,4$.
%Indeed, if it not the case, then curvature at point $a^1$ will be at least $\pi$, which is not possible for convex bodies.
%We also note that for sufficiently large $n$ (when $\varepsilon<\pi/4$) the point $a^1$ should belongs to some of the lune $L_n^i$, because in other case the total curvature of $\Sigma$ will be greater than $4\pi$, which contradicts the Gauss--Bonnet theorem.
%Therefore it could belong only to $L_n^1$.
%The same arguments work for points $a^i$, $i=2,3,4$, and these points are different because they belong to a different $L_n^i$.

\section{Isosceles tetrahedron}

\begin{thm}{Lemma} 
Assume that the closed convex surface $\Sigma$ in $\RR^3$
has $4$ singular points with curvature $\pi$.
Then it bounds an isosceles tetrahedron.
\end{thm}

{

\begin{wrapfigure}{o}{21 mm}
\begin{lpic}[t(-6 mm),b(-3 mm),r(0 mm),l(0 mm)]{pics/akopyan(1)}
\lbl[tr]{3,10;$x$}
\lbl[tl]{14,10;$y$}
\lbl[t]{10.5,0;$z$}
\end{lpic}
\end{wrapfigure}

\parit{Proof.}
Denote the singular points by $p$, $x$, $y$ and $z$.

Let us cut $\Sigma$ along three geodesics $[px]$, $[py]$ and $[pz]$.
Develop the obtained flat surface on the plane.
Since each point $x$, $y$ and $z$ have curvature $\pi$,
the pairs of cuts at these points open straight.
Therefore, the development forms a triangle say $\triangle$ ---
the points $x$, $y$ and $z$ correspond to the midpoints of the sides of $\triangle$
and the point $p$ correspond to the vertices of $\triangle$.

}

It follows that $\Sigma$ is isometric to the surface of isosceles tetrahedron.
The statement now follows from the Pogorelov's uniqueness theorem,
mentioned in the theory overview.

Alternatively one could apply the following exercise and the rigidity theorem (\ref{Rigidity theorem}) which has much simple proof.
\qeds

\begin{thm}{Exercise}
Assume that a convex body $K$ is bounded by a closed flat surface with finite number of singular points.
Show that $K$ is a polyhedron.
\end{thm}



 \bibliographystyle{abbrv}
\bibliography{long-geodesic.bib}{}


\end{document}
